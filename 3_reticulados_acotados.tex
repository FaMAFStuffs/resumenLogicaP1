\section{Reticulados Acotados}

  \PN \textbf{RETICULADO ACOTADO:} Sea $(L, \SU, \IN, 0, 1)$, donde $L \neq \emptyset, \ \SU$ e $\IN$ operaciones
  binarias sobre $L$ y $0, 1 \in L$, será llamada un \textbf{reticulado acotado} si $(L, \SU, \IN)$ es un reticulado y
  además se cumplen las siguientes identidades:
  \begin{enumerate}
    \item[(I8)] $0 \ \SU \ x = x$, para cada $x \in L$
    \item[(I9)] $x \ \SU \ 1 = 1$, para cada $x \in L$
  \end{enumerate}

  \vspace{3mm}
  \PN \textbf{SUBRETICULADO ACOTADO:} Sea $(L, \SU, \IN, 0, 1)$ un reticulado acotado. $S \neq \emptyset \subseteq L$
  será llamado \textbf{subuniverso} de $(L, \SU, \IN, 0, 1)$ si es cerrado bajo las operaciones $\SU$ e $\IN$. Diremos
  que el reticulado acotado $(S, \SU\mathrm{\mid}_{S \times S}, \IN\mathrm{\mid}_{S \times S}, 0, 1)$ es
  \textbf{subreticulado acotado} de $(L, \SU, \IN, 0, 1)$.

  \vspace{3mm}
  \PN \textbf{HOMOMORFISMO E ISOMORFISMO DE RETICULADOS ACOTADOS:} Sean $(L, \SU, \IN, 0, 1)$ y
  $(L^{\prime}, \SU^{\prime}, \IN^{\prime}, 0^{\prime}, 1^{\prime})$ reticulados.
  \begin{itemize}
    \item Una función $F: L \rightarrow L^{\prime}$ será llamada un \textbf{homomorfismo} de $(L, \SU, \IN, 0, 1)$ en
      $(L^{\prime}, \SU^{\prime}, \IN^{\prime}, 0^{\prime}, 1^{\prime})$ si $\forall x, y \in L$ se cumple que:
      \[
        \begin{array}{rcl}
          F(x \mathsf{\;s\;} y) &=& F(x) \ \SU^{\prime} \ F(y) \\
          F(x \mathsf{\;i\;} y) &=& F(x) \ \IN^{\prime} \ F(y) \\
          F(0) &=& 0^{\prime} \\
          F(1) &=& 1^{\prime}
        \end{array}
      \]
    \item Una función $F: L \rightarrow L^{\prime}$ será llamada un \textbf{isomorfismo} de $(L, \SU, \IN, 0, 1)$ en
      $(L^{\prime}, \SU^{\prime}, \IN^{\prime}, 0^{\prime}, 1^{\prime})$ si $F$ es \textbf{biyectiva} y tanto $F$ como
      $F^{-1}$ son \textbf{homomorfismos}.
  \end{itemize}

  \vspace{3mm}
  \PN \textbf{CONGRUENCIAS DE RETICULADOS ACOTADOS:} Sea $(L, \SU, \IN, 0, 1)$ un reticulado, una \textbf{congruencia}
  sobre $(L, \SU, \IN, 0, 1)$ será una \textbf{relación de equivalencia} $\theta$ la cual cumpla:
  \[
    x \theta x^{\prime} \text{ y } y \theta y^{\prime} \Rightarrow (x \ \SU \ y) \theta (x^{\prime} \ \SU \ y^{\prime})
    \text{ y } (x \ \IN \ y) \theta (x^{\prime} \ \IN \ y^{\prime})
  \]

  \PN Definimos, sobre $L/\theta$, $\mathsf{\tilde{s}}$ e $\mathsf{\tilde{\imath}}$, de la siguiente manera:
  \[
    \begin{array}{rcl}
      x/\theta \ \mathsf{\tilde{s}} \ y/\theta &=& (x \ \SU \ y)/\theta \\
      x/\theta \ \mathsf{\tilde{i}} \ y/\theta &=& (x \ \IN \ y)/\theta
    \end{array}
  \]
